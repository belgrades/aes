\subsection{Operations}

\begin{frame}[t]{Prime fields}{Operations}
	When the field $GF(p)$ is a prime field we encounter that:
	
	\begin{itemize}
		\item All nonzero elements of $GF(p)$ have an inverse.
		\item Arithmetic in $GF(p)$ is done modulo $p$. 
	\end{itemize}
	
\end{frame}

\begin{frame}[t]{Prime fields}{Operations example}
	Given $GF(5)$ and $a, b \in GF(5)$
	\medskip
	
	\begin{columns}
		\begin{column}{0.5\textwidth}
			\centering $+(a, b) = a + b \mod 5$				
			\begin{table}[]
				\begin{tabular}{cccccc}
					\textbf{+} & \textbf{0} & \textbf{1} & \textbf{2} & \textbf{3} & \textbf{4} \\
					\textbf{0} & 0          & 1          & 2          & 3          & 4          \\
					\textbf{1} & 1          & 2          & 3          & 4          & 0          \\
					\textbf{2} & 2          & 3          & 4          & 0          & 1          \\
					\textbf{3} & 3          & 4          & 0          & 1          & 2          \\
					\textbf{4} & 4          & 0          & 1          & 2          & 3         
				\end{tabular}
			\end{table}
		\end{column}
		\begin{column}{0.5\textwidth}  %%<--- here
			\centering $\times(a, b) = a * b \mod 5$		
			\begin{table}[]
				\begin{tabular}{cccccc}
					\textbf{$\times$} & \textbf{0} & \textbf{1} & \textbf{2} & \textbf{3} & \textbf{4} \\
					\textbf{0} & 0          & 0          & 0          & 0          & 0          \\
					\textbf{1} & 0          & 1          & 2          & 3          & 4          \\
					\textbf{2} & 0          & 2          & 4          & 1          & 3          \\
					\textbf{3} & 0          & 3          & 1          & 4          & 2          \\
					\textbf{4} & 0          & 4          & 3          & 2          & 1         
				\end{tabular}
			\end{table}
		\end{column}
	\end{columns}
\end{frame}

\begin{frame}[t]{Prime fields}{Additive Inverse}
	Given $GF(5)$ and $a, b \in GF(5)$
	\medskip
	
	\begin{columns}
		\begin{column}{0.5\textwidth}
			\centering $+(a, b) = a + b \mod 5$				
			\begin{table}[]
				\begin{tabular}{cccccc}
					\textbf{+} & \textbf{0} & \textbf{1} & \textbf{2} & \textbf{3} & \textbf{4} \\
					\textbf{0} & 0          & 1          & 2          & 3          & 4          \\
					\textbf{1} & 1          & 2          & 3          & 4          & 0          \\
					\textbf{2} & 2          & 3          & 4          & 0          & 1          \\
					\textbf{3} & 3          & 4          & 0          & 1          & 2          \\
					\textbf{4} & 4          & 0          & 1          & 2          & 3         
				\end{tabular}
			\end{table}
		\end{column}
		\begin{column}{0.5\textwidth}  %%<--- here
			\centering $-a$		
			\begin{table}[]
				\begin{tabular}{ccc}
					-0  & = & 1 \\
					-1 & = & 4 \\
					-2  & = & 3 \\
					-3  & = & 2 \\
					-4  & = & 1
				\end{tabular}
			\end{table}
		\end{column}
	\end{columns}
\end{frame}

\begin{frame}[t]{Prime fields}{Multiplicative inverse}
	Given $GF(5)$ and $a, b \in GF(5)$
	\medskip
	
	\begin{columns}
		\begin{column}{0.5\textwidth}
			\centering $\times(a, b) = a * b \mod 5$		
			\begin{table}[]
				\begin{tabular}{cccccc}
					\textbf{$\times$} & \textbf{0} & \textbf{1} & \textbf{2} & \textbf{3} & \textbf{4} \\
					\textbf{0} & 0          & 0          & 0          & 0          & 0          \\
					\textbf{1} & 0          & 1          & 2          & 3          & 4          \\
					\textbf{2} & 0          & 2          & 4          & 1          & 3          \\
					\textbf{3} & 0          & 3          & 1          & 4          & 2          \\
					\textbf{4} & 0          & 4          & 3          & 2          & 1         
				\end{tabular}
			\end{table}
		\end{column}
		\begin{column}{0.5\textwidth}  %%<--- here
			\centering $a^{-1}$		
			\begin{table}[]
				\begin{tabular}{ccc}
					$0^{-1}$  & = & Doesn't exist \\
					$1^{-1}$  & = & 1 \\
					$2^{-1}$  & = & 3 \\
					$3^{-1}$  & = & 2 \\
					$4^{-1}$  & = & 4
				\end{tabular}
			\end{table}
		\end{column}
	\end{columns}
\end{frame}

\begin{frame}[t]{Sbox}
	\begin{table}[]
		\tiny 
		\setlength{\tabcolsep}{2pt} 
		\begin{tabular}{c|cccccccccccccccc}
			& 0    &    1 &    2 &    3 &    4 &    5 &    6 &    7 &    8 &    9 &    A &    B &  C &   D  &  E &  F \\ \hline
			0 & 0x63 & 0x7C & 0x77 & 0x7B & 0xF2 & 0x6B & 0x6F & 0xC5 & 0x30 & 0x01 & 0x67 & 0x2B & 0xFE & 0xD7 & 0xAB & 0x76 \\
			1 & 0xCA & 0x82 & 0xC9 & 0x7D & 0xFA & 0x59 & 0x47 & 0xF0 & 0xAD & 0xD4 & 0xA2 & 0xAF & 0x9C & 0xA4 & 0x72 & 0xC0 \\
			2 & 0xB7 & 0xFD & 0x93 & 0x26 & 0x36 & 0x3F & 0xF7 & 0xCC & 0x34 & 0xA5 & 0xE5 & 0xF1 & 0x71 & 0xD8 & 0x31 & 0x15 \\
			3 & 0x04 & 0xC7 & 0x23 & 0xC3 & 0x18 & 0x96 & 0x05 & 0x9A & 0x07 & 0x12 & 0x80 & 0xE2 & 0xEB & 0x27 & 0xB2 & 0x75 \\
			4 & 0x09 & 0x83 & 0x2C & 0x1A & 0x1B & 0x6E & 0x5A & 0xA0 & 0x52 & 0x3B & 0xD6 & 0xB3 & 0x29 & 0xE3 & 0x2F & 0x84 \\
			5 & 0x53 & 0xD1 & 0x00 & 0xED & 0x20 & 0xFC & 0xB1 & 0x5B & 0x6A & 0xCB & 0xBE & 0x39 & 0x4A & 0x4C & 0x58 & 0xCF \\
			6 & 0xD0 & 0xEF & 0xAA & 0xFB & 0x43 & 0x4D & 0x33 & 0x85 & 0x45 & 0xF9 & 0x02 & 0x7F & 0x50 & 0x3C & 0x9F & 0xA8 \\
			7 & 0x51 & 0xA3 & 0x40 & 0x8F & 0x92 & 0x9D & 0x38 & 0xF5 & 0xBC & 0xB6 & 0xDA & 0x21 & 0x10 & 0xFF & 0xF3 & 0xD2 \\
			8 & 0xCD & 0x0C & 0x13 & 0xEC & 0x5F & 0x97 & 0x44 & 0x17 & 0xC4 & 0xA7 & 0x7E & 0x3D & 0x64 & 0x5D & 0x19 & 0x73 \\
			9 & 0x60 & 0x81 & 0x4F & 0xDC & 0x22 & 0x2A & 0x90 & 0x88 & 0x46 & 0xEE & 0xB8 & 0x14 & 0xDE & 0x5E & 0x0B & 0xDB \\
			A & 0xE0 & 0x32 & 0x3A & 0x0A & 0x49 & 0x06 & 0x24 & 0x5C & 0xC2 & 0xD3 & 0xAC & 0x62 & 0x91 & 0x95 & 0xE4 & 0x79 \\
			B & 0xE7 & 0xC8 & 0x37 & 0x6D & 0x8D & 0xD5 & 0x4E & 0xA9 & 0x6C & 0x56 & 0xF4 & 0xEA & 0x65 & 0x7A & 0xAE & 0x08 \\
			C & 0xBA & 0x78 & 0x25 & 0x2E & 0x1C & 0xA6 & 0xB4 & 0xC6 & 0xE8 & 0xDD & 0x74 & 0x1F & 0x4B & 0xBD & 0x8B & 0x8A \\
			D & 0x70 & 0x3E & 0xB5 & 0x66 & 0x48 & 0x03 & 0xF6 & 0x0E & 0x61 & 0x35 & 0x57 & 0xB9 & 0x86 & 0xC1 & 0x1D & 0x9E \\
			E & 0xE1 & 0xF8 & 0x98 & 0x11 & 0x69 & 0xD9 & 0x8E & 0x94 & 0x9B & 0x1E & 0x87 & 0xE9 & 0xCE & 0x55 & 0x28 & 0xDF \\
			F & 0x8C & 0xA1 & 0x89 & 0x0D & 0xBF & 0xE6 & 0x42 & 0x68 & 0x41 & 0x99 & 0x2D & 0x0F & 0xB0 & 0x54 & 0xBB & 0x16 
		\end{tabular}
	\end{table}
\end{frame}

\begin{frame}[t]{Prime fields}{Multiplicative inverse}
	Given $GF(2)$ and $a, b \in GF(2)$ 
	\bigskip
	\medskip	
	\begin{columns}
		\begin{column}{0.5\textwidth}
			\centering $\oplus(a, b) = a * b \mod 2$		
			\begin{table}[]
				\begin{tabular}{c|cc}
					\textbf{$\oplus$} & \textbf{0} & \textbf{1} \\ \hline
					\textbf{0} & 0         & 1           \\
					\textbf{1} &1          & 0           \\
					      
				\end{tabular}
			\end{table}
		\end{column}
		\begin{column}{0.5\textwidth}  %%<--- here
			\centering $\wedge(a, b) = a * b \mod 2$	
			\begin{table}[]
				\begin{tabular}{c|cc}
					\textbf{$\wedge$} & \textbf{0} & \textbf{1} \\ \hline
					\textbf{0} & 0          & 0                 \\
					\textbf{1} & 0          & 1                 \\
					        
				\end{tabular}
			\end{table}
		\end{column}
	\end{columns}
\end{frame}

\begin{frame}[t]{Prime fields}{Bytes}
	All byte values in the \textbf{AES} algorithm will be presented as the concatenation of its individual bit values (0 or 1) between braces in the order $\{ b_7, b_6, b_5, b_4, b_3, b_2, b_1, b_0 \}$. These bytes are interpreted as finite field elements using a polynomial representation: \bigskip
	

	$b_7x^7 + b_6x^6 + b_5x^5 + b_4x^4 + b_3x^3 + b_2x^2 + b_1x^1 + b_0 = \sum_{i=0}^{7}b_ix^i$. \\[10pt]


	For example, $\{01100011\}$ identifies the specific finite field element $x^6 + x^5 + x + 1$.
		
\end{frame}


\begin{frame}[t]{Prime fields}{Bytes}
	It is also convenient to denote byte values using hexadecimal notation with each of two groups of four bits being denoted by a single character as followed: \\[5pt]

	\begin{columns}
		\tiny 
		\setlength{\tabcolsep}{3pt} 
		\begin{column}{0.5\textwidth}		
			\begin{table}[]
				\begin{tabular}{c|c}
					\textbf{Bit Pattern} & \textbf{Character} \\ \hline
					\textbf{0000} & 0                  \\
					\textbf{0001} & 1                  \\
					\textbf{0010} & 2                  \\
					\textbf{0011} & 3                  \\
					\textbf{0100} & 4                  \\
					\textbf{0101} & 5                  \\
					\textbf{0110} & 6                  \\
					\textbf{0111} & 7                  \\					
				\end{tabular}
			\end{table}
		\end{column}

		
		\begin{column}{0.5\textwidth}  %%<--- here	
			\begin{table}[]
				\begin{tabular}{c|c}
					\textbf{Bit Pattern} & \textbf{Character} \\ \hline
					\textbf{1000} & 8                  \\
					\textbf{1001} & 9                  \\
					\textbf{1010} & A                  \\
					\textbf{1011} & B                  \\	
					\textbf{1100} & C                  \\
					\textbf{1101} & D                  \\
					\textbf{1110} & E                  \\
					\textbf{1111} & F                  \\
					
				\end{tabular}
			\end{table}
		\end{column}
	\end{columns}

	\centering Figure 1. Hexadecimal representation of bit patterns.	
\end{frame}


\begin{frame}[t]{Prime fields}{Addition Example}
	
	For example, the following expressions are equivalent to one another: 
	\medskip
	
	(polynomial notation);
	$(x^6 + x^4 + x^2 + x + 1) + (x^7 + x + 1) = x^7 + x^6 + x^4 + x^2$ 
	
	\bigskip
	
	(binary notation); \\
	$\{01010111\} \wedge \{10000011\} = \{11010100\}$
	
	\bigskip
	
	(hexadecimal notation);	\\	
	$\{57\} \wedge \{83\} = \{D4\}$ 	


\end{frame}

\begin{frame}[t]{Prime fields}{Example}
	
	For example, $\{57\} \bullet \{83\} = \{C1\}$, because 
	\medskip
	

	\begin{center}
		\small
		\begin{tabular}{ccl}\
			$(x^6 + x^4 + x^2 + x + 1)(x^7 + x + 1)$ & = &  $x^{13} + x^{11} + x^9 + x^8 + x^7 + x^7 + $ \\
			& & $ x^5 + x^3 + x^2 + x + x^6 + x^4 + x^2 $ \\
			& & $+ x + 1$ \\
			& = & $x^{13} + x^{11} + x^9 + x^8 + x^6 + x^5 + $ \\
			& & $ x^4 + x^3 + 1$\\
			
	\end{tabular}\end{center}
	
	and
	
	\begin{center}
		\small
			$x^{13} + x^{11} + x^9 + x^8 + x^6 + x^5 + x^4 + x^3 + 1 \text{ modulo} (x^8 + x^4 + x^3 + x + 1) $ \\	
			$ = x^{7} + x^6 +1 $ \\
			
	\end{center}
\end{frame}


\begin{frame}[t]{Prime fields}{Multiplication}
	For any non-zero binary polynomial $b(x)$ of degree less than 8, the multiplicative inverse of $b(x)$, denoted $b^{-1}(x)$, can be found as follows: the extended Euclidean algorithm is used to compute polynomials $a(x)$ and $c(x)$ such that \\
	\medskip
	
	\centering{ $b(x)a(x) + m(x)c(x) = 1$}
	
	\medskip
	
	\begin{flushleft}
		Hence, $a(x) \bullet b(x) \text{mod} m(x) = 1$, which means
	\end{flushleft}

	
	\centering{ $b{-1}(x) = a(x) \text{mod} m(x)$.}
	
	\medskip
	\begin{flushleft}
		Moreover, for any $a(X) \text{,} b(x) \text{and} c(x)$ in the field, it holds that
	\end{flushleft}
	
	\centering{ $a(x) \bullet (b(x)+c(x)) = a(x) \bullet b(x) + a(x) \bullet c(x)$}
	
\end{frame}

\begin{frame}[t]{Prime fields}{Multiplication by x}
	Multiplying the binary polynomial previously defined with the polynomial $x$ results in \\
	\medskip
	
	\centering{ $b_7x^8 + b_6x^7 + b_5x^6 + b_4x^5 + b_3x^4 + b_2x^3 + b_1x^2 + b_0x$.}
	
	\medskip
	
	\begin{flushleft}
		The result $x \bullet b(x)$ is obtained by reducing the above result modulo $m(x)$, as defined in \cite{Rijndael2020design}. If $b_7 = 0$, the result is already in reduced form. If $b_7=1$, the reduction is accomplished by subtracting (i.e., XORing) the polynomial $m(x)$. It follows  that multiplication by $x$ (i.e., $\{00000010\}$ or $\{02\}$) can be implemented at the byte level as a left shift and a subsequent conditional bit wise XOR with $\{1B\}$. This operations on bytes is denoted by $xtime()$.
	\end{flushleft}

\end{frame}



\begin{frame}[t]{Prime fields}{Other Example}
	
	For example, $\{57\} \bullet \{13\} = \{FE\}$, because 
	\medskip
	
	
	\begin{center}
		\small
		\begin{tabular}{ccccccl}\
			$\{57\}$ & $\bullet$ & $\{02\}$ & = &  $xtime(\{57\})$ & = & $\{AE\}$ \\
			$\{57\}$ & $\bullet$ & $\{04\}$ & = &  $xtime(\{AE\})$ & = & $\{47\}$ \\
			$\{57\}$ & $\bullet$ & $\{08\}$ & = &  $xtime(\{47\})$ & = & $\{8E\}$ \\
			$\{57\}$ & $\bullet$ & $\{10\}$ & = &  $xtime(\{8E\})$ & = & $\{07\}$ \\

	\end{tabular}\end{center}
	
	thus,
	
	\begin{center}
		\small
		\begin{tabular}{ccl}\
			$\{57\} \bullet \{13\}$ & = &  $\{57\} \bullet (\{01\} \oplus \{02\} \oplus \{10\})$ \\
			& = & $\{57\} \oplus \{AE\} \oplus \{07\}$ \\
			& = &  $FE$ \\
			
	\end{tabular}\end{center}
\end{frame}
