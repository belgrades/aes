\subsection{Operations}

\begin{frame}[t]{Prime fields}{Operations}
	When the field $GF(p)$ is a prime field we encounter that:
	
	\begin{itemize}
		\item All nonzero elements of $GF(p)$ have an inverse.
		\item Arithmetic in $GF(p)$ is done modulo $p$. 
	\end{itemize}
	
\end{frame}

\begin{frame}[t]{Prime fields}{Operations example}
	Given $GF(5)$ and $a, b \in GF(5)$
	\medskip
	
	\begin{columns}
		\begin{column}{0.5\textwidth}
			\centering $+(a, b) = a + b \mod 5$				
			\begin{table}[]
				\begin{tabular}{c|ccccc}
					\textbf{+} & \textbf{0} & \textbf{1} & \textbf{2} & \textbf{3} & \textbf{4} \\ \hline
					\textbf{0} & 0          & 1          & 2          & 3          & 4          \\
					\textbf{1} & 1          & 2          & 3          & 4          & 0          \\
					\textbf{2} & 2          & 3          & 4          & 0          & 1          \\
					\textbf{3} & 3          & 4          & 0          & 1          & 2          \\
					\textbf{4} & 4          & 0          & 1          & 2          & 3         
				\end{tabular}
			\end{table}
		\end{column}
		\begin{column}{0.5\textwidth}  %%<--- here
			\centering $\times(a, b) = a * b \mod 5$		
			\begin{table}[]
				\begin{tabular}{c|ccccc}
					\textbf{$\times$} & \textbf{0} & \textbf{1} & \textbf{2} & \textbf{3} & \textbf{4} \\ \hline
					\textbf{0} & 0          & 0          & 0          & 0          & 0          \\ 
					\textbf{1} & 0          & 1          & 2          & 3          & 4          \\
					\textbf{2} & 0          & 2          & 4          & 1          & 3          \\
					\textbf{3} & 0          & 3          & 1          & 4          & 2          \\
					\textbf{4} & 0          & 4          & 3          & 2          & 1         
				\end{tabular}
			\end{table}
		\end{column}
	\end{columns}
\end{frame}

\begin{frame}[t]{Prime fields}{Additive Inverse}
	Given $GF(5)$ and $a, b \in GF(5)$
	\medskip
	
	\begin{columns}
		\begin{column}{0.5\textwidth}
			\centering $+(a, b) = a + b \mod 5$				
			\begin{table}[]
				\begin{tabular}{c|ccccc}
					\textbf{+} & \textbf{0} & \textbf{1} & \textbf{2} & \textbf{3} & \textbf{4} \\ \hline
					\textbf{0} & 0          & 1          & 2          & 3          & 4          \\
					\textbf{1} & 1          & 2          & 3          & 4          & 0          \\
					\textbf{2} & 2          & 3          & 4          & 0          & 1          \\
					\textbf{3} & 3          & 4          & 0          & 1          & 2          \\
					\textbf{4} & 4          & 0          & 1          & 2          & 3         
				\end{tabular}
			\end{table}
		\end{column}
		\begin{column}{0.5\textwidth}  %%<--- here
			\centering $-a$		
			\begin{table}[]
				\begin{tabular}{c|cc}
					-0  & = & 1 \\ \hline
					-1 & = & 4 \\
					-2  & = & 3 \\
					-3  & = & 2 \\
					-4  & = & 1
				\end{tabular}
			\end{table}
		\end{column}
	\end{columns}
\end{frame}

\begin{frame}[t]{Prime fields}{Multiplicative inverse}
	Given $GF(5)$ and $a, b \in GF(5)$
	\medskip
	
	\begin{columns}
		\begin{column}{0.5\textwidth}
			\centering $\times(a, b) = a * b \mod 5$		
			\begin{table}[]
				\begin{tabular}{c|ccccc}
					\textbf{$\times$} & \textbf{0} & \textbf{1} & \textbf{2} & \textbf{3} & \textbf{4} \\ \hline
					\textbf{0} & 0          & 0          & 0          & 0          & 0          \\
					\textbf{1} & 0          & 1          & 2          & 3          & 4          \\
					\textbf{2} & 0          & 2          & 4          & 1          & 3          \\
					\textbf{3} & 0          & 3          & 1          & 4          & 2          \\
					\textbf{4} & 0          & 4          & 3          & 2          & 1         
				\end{tabular}
			\end{table}
		\end{column}
		\begin{column}{0.5\textwidth}  %%<--- here
			\centering $a^{-1}$		
			\begin{table}[]
				\begin{tabular}{c|cc}
					$0^{-1}$  & = & Doesn't exist \\ \hline
					$1^{-1}$  & = & 1 \\
					$2^{-1}$  & = & 3 \\
					$3^{-1}$  & = & 2 \\
					$4^{-1}$  & = & 4
				\end{tabular}
			\end{table}
		\end{column}
	\end{columns}
\end{frame}

\begin{frame}[t]{Prime fields}{GF(2)}
	Given $GF(2)$ and $a, b \in GF(2)$ 
	\bigskip
	\medskip	
	\begin{columns}
		\begin{column}{0.5\textwidth}
			\centering $\oplus(a, b) = a + b \mod 2$		
			\begin{table}[]
				\begin{tabular}{c|cc}
					\textbf{$\oplus$} & \textbf{0} & \textbf{1} \\ \hline
					\textbf{0} & 0         & 1           \\
					\textbf{1} &1          & 0           \\
					      
				\end{tabular}
			\end{table}
		\end{column}
		\begin{column}{0.5\textwidth}  %%<--- here
			\centering $\wedge(a, b) = a * b \mod 2$	
			\begin{table}[]
				\begin{tabular}{c|cc}
					\textbf{$\wedge$} & \textbf{0} & \textbf{1} \\ \hline
					\textbf{0} & 0          & 0                 \\
					\textbf{1} & 0          & 1                 \\
					        
				\end{tabular}
			\end{table}
		\end{column}
	\end{columns}
\end{frame}

\begin{frame}[t]{Extension Fields}{$GF(2^m)$}
	When m is different than 1. The operations regarding our fields change. In the case of the AES, and for convenient reasons, m is equal to 8.
	
	A field $F$ defined as $GF(2^m)$ will be called an extension field.
	
	Before writing the definition of the multiplication operation over $G(2^8)$ let's introduce the notion of polynomials with coefficients in $G(2^8)$.
	
\end{frame}

\begin{frame}[t]{Polynomials with coefficients in $G(2^8)$}{Definition}
	In the \textbf{AES} algorithm will be presented as the concatenation of its individual bit values (0 or 1) between braces in the order $\{ b_7, b_6, b_5, b_4, b_3, b_2, b_1, b_0 \}$. As we have 8 bits, we can store it exactly in 1 byte of memory. Let's write these elements using a polynomial representation: \bigskip
	

	$b_7x^7 + b_6x^6 + b_5x^5 + b_4x^4 + b_3x^3 + b_2x^2 + b_1x^1 + b_0 = \sum_{i=0}^{7}b_ix^i$. \\[10pt]


	For example, $\{01100011\}$ identifies the specific finite field element $x^6 + x^5 + x + 1$.
		
\end{frame}

\begin{frame}[t]{Polynomials with coefficients in $G(2^8)$}{Representation}
	It is also convenient to denote byte values using hexadecimal notation with each of two groups of four bits being denoted by a single character as followed: \\[5pt]

	\begin{columns}
		\tiny 
		\setlength{\tabcolsep}{3pt} 
		\begin{column}{0.5\textwidth}		
			\begin{table}[]
				\begin{tabular}{c|c}
					\textbf{Bit Pattern} & \textbf{Character} \\ \hline
					\textbf{0000} & 0                  \\
					\textbf{0001} & 1                  \\
					\textbf{0010} & 2                  \\
					\textbf{0011} & 3                  \\
					\textbf{0100} & 4                  \\
					\textbf{0101} & 5                  \\
					\textbf{0110} & 6                  \\
					\textbf{0111} & 7                  \\					
				\end{tabular}
			\end{table}
		\end{column}

		
		\begin{column}{0.5\textwidth}  %%<--- here	
			\begin{table}[]
				\begin{tabular}{c|c}
					\textbf{Bit Pattern} & \textbf{Character} \\ \hline
					\textbf{1000} & 8                  \\
					\textbf{1001} & 9                  \\
					\textbf{1010} & A                  \\
					\textbf{1011} & B                  \\	
					\textbf{1100} & C                  \\
					\textbf{1101} & D                  \\
					\textbf{1110} & E                  \\
					\textbf{1111} & F                  \\
					
				\end{tabular}
			\end{table}
		\end{column}
	\end{columns}

	\centering Figure 1. Hexadecimal representation of bit patterns.	
\end{frame}


\begin{frame}[t]{Polynomials with coefficients in $G(2^8)$}{Example}
	
	For example, the following expressions are equivalent to one another: 
	\medskip
	
	(polynomial notation);
	$(x^6 + x^4 + x^2 + x + 1) + (x^7 + x + 1) = x^7 + x^6 + x^4 + x^2$ 
	
	\bigskip
	
	(binary notation); \\
	$\{01010111\} \wedge \{10000011\} = \{11010100\}$
	
	\bigskip
	
	(hexadecimal notation);	\\	
	$\{57\} \wedge \{83\} = \{D4\}$ 	


\end{frame}

\begin{frame}[t]{Polynomials with coefficients in $G(2^8)$}{Multiplication}
	In the polynomial representation, multiplication in $G(2^8)$ (denoted as $bullet$) corresponds to:
	
	\medskip
	
	Let $a(x), b(x) \in GF(2^8)$ 
	
	\begin{enumerate}[i.]
		\item Compute $a(x) * b(x)$ as in normal polynomial calculus.
		\item Apply the modulo using an \textbf{irreducible polynomial} $m(x)$ of degree $8$.
	
	\end{enumerate}
	\begin{itemize}
		\item A polynomial is irreducible if its only divisors are $1$ and itself.
		\item Doing this, we ensure that $a(x) \bullet b(x) \in GF(2^8)$.
		\item In the AES, $m(x) = x^8 + x^4 + x^3 + x + 1$
	\end{itemize}
	
\end{frame}

\begin{frame}[t]{Polynomials with coefficients in $G(2^8)$}{Multiplication example}
	
	For example, $\{57\} \bullet \{83\} = \{C1\}$, because 
	\medskip
	

	\begin{center}
		\small
		\begin{tabular}{ccl}\
			$(x^6 + x^4 + x^2 + x + 1)(x^7 + x + 1)$ & = &  $x^{13} + x^{11} + x^9 + x^8 + x^7 + x^7 + $ \\
			& & $ x^5 + x^3 + x^2 + x + x^6 + x^4 + x^2 $ \\
			& & $+ x + 1$ \\
			& = & $x^{13} + x^{11} + x^9 + x^8 + x^6 + x^5 + $ \\
			& & $ x^4 + x^3 + 1$\\
			
	\end{tabular}\end{center}
	
	and
	
	\begin{center}
		\small
			$x^{13} + x^{11} + x^9 + x^8 + x^6 + x^5 + x^4 + x^3 + 1 \text{ modulo} (x^8 + x^4 + x^3 + x + 1) $ \\	
			$ = x^{7} + x^6 +1 $ \\
			
	\end{center}
\end{frame}

\begin{frame}[t]{Polynomials with coefficients in $G(2^8)$}{The xtime operation}
	Multiplying the binary polynomial previously defined with the polynomial $x$ results in \\
	\medskip
	
	\centering{ $b_7x^8 + b_6x^7 + b_5x^6 + b_4x^5 + b_3x^4 + b_2x^3 + b_1x^2 + b_0x$.}
	
	\medskip
	
	\begin{itemize}
		\item $x \bullet b(x)$ is obtained by reducing the above result modulo $m(x)$.
		\item If $b_7 = 0$, the result is already in reduced form.
		\item If $b_7=1$, the reduction is accomplished by subtracting the polynomial $m(x)$.
		\item It follows  that multiplication by $x$ ($\{02\}$) can be implemented at the byte level as a left shift and a subsequent conditional bit wise XOR with $\{1B\}$. 
	\end{itemize}
	
\end{frame}


\begin{frame}[t]{Polynomials with coefficients in $G(2^8)$}{Other Example}
	
	For example, $\{57\} \bullet \{13\} = \{FE\}$, because 
	\medskip
	
	
	\begin{center}
		\small
		\begin{tabular}{ccccccl}
			
			$\{57\}$ & $\bullet$ & $\{02\}$ & = &  $xtime(\{57\})$ & = & $\{AE\}$ \\
			$\{57\}$ & $\bullet$ & $\{04\}$ & = &  $xtime(\{AE\})$ & = & $\{47\}$ \\
			$\{57\}$ & $\bullet$ & $\{08\}$ & = &  $xtime(\{47\})$ & = & $\{8E\}$ \\
			$\{57\}$ & $\bullet$ & $\{10\}$ & = &  $xtime(\{8E\})$ & = & $\{07\}$ \\
			
	\end{tabular}\end{center}
	
	thus,
	
	\begin{center}
		\small
		\begin{tabular}{ccl}\
			$\{57\} \bullet \{13\}$ & = &  $\{57\} \bullet (\{01\} \oplus \{02\} \oplus \{10\})$ \\
			& = & $\{57\} \oplus \{AE\} \oplus \{07\}$ \\
			& = &  $FE$ \\
			
	\end{tabular}\end{center}
\end{frame}


\begin{frame}[t]{Polynomials with coefficients in $G(2^8)$}{Inversion in $GF(2^8)$}
	For any non-zero binary polynomial $b(x)$ of degree less than 8, the multiplicative inverse of $b(x)$, denoted $b^{-1}(x)$, can be found as follows: the extended Euclidean algorithm is used to compute polynomials $a(x)$ and $c(x)$ such that \\
	\medskip
	
	\centering{ $b(x)a(x) + m(x)c(x) = 1$}
	
	\medskip
	
	\begin{flushleft}
		Hence, $a(x) \bullet b(x) \mod m(x) = 1$, which means
	\end{flushleft}

	
	\centering{ $b^{-1}(x) = a(x) \mod  m(x)$.}
	
	\medskip
	\begin{flushleft}
		Moreover, for any $a(X) \text{,} b(x) \text{ and } c(x)$ in the field, it holds that
	\end{flushleft}
	
	\centering{ $a(x) \bullet (b(x)+c(x)) = a(x) \bullet b(x) + a(x) \bullet c(x)$}
	
\end{frame}

\section{The SubBytes and InvSubBytes Operations}

\subsection{The SubBytes}

\begin{frame}[t]{SubBytes}{Definition}
	\begin{itemize}
		\item Non-linear byte substitution that operates independently on each byte of the State.
		\item Constructed by composing two transformations
			\begin{enumerate}
				\item Take the multiplicative inverse in the finite field GF($2^8$); the element $\{00\}$ is mapped to itself.
				\item Apply the following affine transformation (over GF(2)): 
				\begin{center}
					$b_i^{'}= b_i \oplus b_{(i+4)mod8} \oplus b_{(i+5)mod8} \oplus b_{(i+6)mod8} \oplus b_{(i+7)mod8} \oplus c_i$
				\end{center}					
				for $0 \leq i < 8$, where $b_i$ is the $i^{th}$ bit of the byte, and $c_i$ is the $i^{th}$ bit of a byte. $c$ with the value $\{63\}$ or $\{01100011\}$
			\end{enumerate}
		\item Is invertible.
	\end{itemize}
	
\end{frame}

\begin{frame}[t]{SubBytes}{The Affine transformation}
	
	In matrix form, the affine transformation element of the S-box can be expressed as:
	
	\[
	\begin{bmatrix}
	b_0^{'} \\
	b_1^{'} \\
	b_2^{'} \\
	b_3^{'} \\
	b_4^{'} \\
	b_5^{'} \\
	b_6^{'} \\
	b_7^{'} \\	
	\end{bmatrix}
	=
	\begin{bmatrix}
	1 & 0 & 0 & 0 & 1 & 1 & 1 & 1\\
	1 & 1 & 0 & 0 & 0 & 1 & 1 & 1\\
	1 & 1 & 1 & 0 & 0 & 0 & 1 & 1\\
	1 & 1 & 1 & 1 & 0 & 0 & 0 & 1\\
	1 & 1 & 1 & 1 & 1 & 0 & 0 & 0\\
	0 & 1 & 1 & 1 & 1 & 1 & 0 & 0\\
	0 & 0 & 1 & 1 & 1 & 1 & 1 & 0\\
	0 & 0 & 0 & 1 & 1 & 1 & 1 & 1\\

	\end{bmatrix}
	\begin{bmatrix}
	b_0 \\
	b_1 \\
	b_2 \\
	b_3 \\
	b_4 \\
	b_5 \\
	b_6 \\
	b_7 \\	
	\end{bmatrix}
	+
	\begin{bmatrix}
	1 \\
	1 \\
	0 \\
	0 \\
	0 \\
	1 \\
	1 \\
	0 \\	
	\end{bmatrix}
	\]

\end{frame}

\begin{frame}[t]{SubBytes}{S-Box Figure}
	
	The following figure illustrates the effect of the \textbf{SubBytes()} transformation on the State:
	
	\medspace
		
	\begin{center}
		\includegraphics[scale=0.8]{images/s_box}
	\end{center}
	
\end{frame}

\begin{frame}[t]{Sbox}
	\small
	The S-box used in the \textbf{SubBytes()} transformation is presented in hexadecimal for as in figure. \\
	For example, if $s_{1,1} = {53}$, then the substitution value would be determined by the intersection of the row with index '5' and the column with index '3'. This would result in $s^{'}_{1,1}$ having a value of ${ED}$.
	
	\begin{table}[]
		\tiny 
		\setlength{\tabcolsep}{2pt} 
		\begin{tabular}{c|cccccccccccccccc}
			& 0    &    1 &    2 &    3 &    4 &    5 &    6 &    7 &    8 &    9 &    A &    B &  C &   D  &  E &  F \\ \hline
			0 & 0x63 & 0x7C & 0x77 & 0x7B & 0xF2 & 0x6B & 0x6F & 0xC5 & 0x30 & 0x01 & 0x67 & 0x2B & 0xFE & 0xD7 & 0xAB & 0x76 \\
			1 & 0xCA & 0x82 & 0xC9 & 0x7D & 0xFA & 0x59 & 0x47 & 0xF0 & 0xAD & 0xD4 & 0xA2 & 0xAF & 0x9C & 0xA4 & 0x72 & 0xC0 \\
			2 & 0xB7 & 0xFD & 0x93 & 0x26 & 0x36 & 0x3F & 0xF7 & 0xCC & 0x34 & 0xA5 & 0xE5 & 0xF1 & 0x71 & 0xD8 & 0x31 & 0x15 \\
			3 & 0x04 & 0xC7 & 0x23 & 0xC3 & 0x18 & 0x96 & 0x05 & 0x9A & 0x07 & 0x12 & 0x80 & 0xE2 & 0xEB & 0x27 & 0xB2 & 0x75 \\
			4 & 0x09 & 0x83 & 0x2C & 0x1A & 0x1B & 0x6E & 0x5A & 0xA0 & 0x52 & 0x3B & 0xD6 & 0xB3 & 0x29 & 0xE3 & 0x2F & 0x84 \\
			5 & 0x53 & 0xD1 & 0x00 & 0xED & 0x20 & 0xFC & 0xB1 & 0x5B & 0x6A & 0xCB & 0xBE & 0x39 & 0x4A & 0x4C & 0x58 & 0xCF \\
			6 & 0xD0 & 0xEF & 0xAA & 0xFB & 0x43 & 0x4D & 0x33 & 0x85 & 0x45 & 0xF9 & 0x02 & 0x7F & 0x50 & 0x3C & 0x9F & 0xA8 \\
			7 & 0x51 & 0xA3 & 0x40 & 0x8F & 0x92 & 0x9D & 0x38 & 0xF5 & 0xBC & 0xB6 & 0xDA & 0x21 & 0x10 & 0xFF & 0xF3 & 0xD2 \\
			8 & 0xCD & 0x0C & 0x13 & 0xEC & 0x5F & 0x97 & 0x44 & 0x17 & 0xC4 & 0xA7 & 0x7E & 0x3D & 0x64 & 0x5D & 0x19 & 0x73 \\
			9 & 0x60 & 0x81 & 0x4F & 0xDC & 0x22 & 0x2A & 0x90 & 0x88 & 0x46 & 0xEE & 0xB8 & 0x14 & 0xDE & 0x5E & 0x0B & 0xDB \\
			A & 0xE0 & 0x32 & 0x3A & 0x0A & 0x49 & 0x06 & 0x24 & 0x5C & 0xC2 & 0xD3 & 0xAC & 0x62 & 0x91 & 0x95 & 0xE4 & 0x79 \\
			B & 0xE7 & 0xC8 & 0x37 & 0x6D & 0x8D & 0xD5 & 0x4E & 0xA9 & 0x6C & 0x56 & 0xF4 & 0xEA & 0x65 & 0x7A & 0xAE & 0x08 \\
			C & 0xBA & 0x78 & 0x25 & 0x2E & 0x1C & 0xA6 & 0xB4 & 0xC6 & 0xE8 & 0xDD & 0x74 & 0x1F & 0x4B & 0xBD & 0x8B & 0x8A \\
			D & 0x70 & 0x3E & 0xB5 & 0x66 & 0x48 & 0x03 & 0xF6 & 0x0E & 0x61 & 0x35 & 0x57 & 0xB9 & 0x86 & 0xC1 & 0x1D & 0x9E \\
			E & 0xE1 & 0xF8 & 0x98 & 0x11 & 0x69 & 0xD9 & 0x8E & 0x94 & 0x9B & 0x1E & 0x87 & 0xE9 & 0xCE & 0x55 & 0x28 & 0xDF \\
			F & 0x8C & 0xA1 & 0x89 & 0x0D & 0xBF & 0xE6 & 0x42 & 0x68 & 0x41 & 0x99 & 0x2D & 0x0F & 0xB0 & 0x54 & 0xBB & 0x16 
		\end{tabular}
	\end{table}
\end{frame}

\subsection{The InvSubBytes}

\begin{frame}[t]{InvSubBytes}{Definition}
	\begin{itemize}
		\item The S-Box can be inversed by applying the inverse of each operation in reverse order.
		\item In the S-Box computation we first applied the Galois Field inverse and then the affine function.
		\item We will apply the inverse of the affine function (which is also affine) and then the inverse of the Galois field which is the function itself.
		\item The derivations of the inverse of the affine function are vaguely specified in the bibliography consulted.
		\item As in the SubBytes operation, we will try to build a lookup table.
	\end{itemize}
	
\end{frame}

\begin{frame}[t]{InvSubBytes}{The inverse Affine transformation}
	
	In matrix form, the affine transformation element of the S-box can be expressed as:
	
	\[
	\begin{bmatrix}
	b_0^{'} \\
	b_1^{'} \\
	b_2^{'} \\
	b_3^{'} \\
	b_4^{'} \\
	b_5^{'} \\
	b_6^{'} \\
	b_7^{'} \\	
	\end{bmatrix}
	=
	\begin{bmatrix}
	0 & 0 & 1 & 0 & 0 & 1 & 0 & 1\\
	1 & 0 & 0 & 1 & 0 & 0 & 1 & 0\\
	0 & 1 & 0 & 0 & 1 & 0 & 0 & 1\\
	1 & 0 & 1 & 0 & 0 & 1 & 0 & 0\\
	0 & 1 & 0 & 1 & 0 & 0 & 1 & 0\\
	0 & 0 & 1 & 0 & 1 & 0 & 0 & 1\\
	1 & 0 & 0 & 1 & 0 & 1 & 0 & 0\\
	0 & 1 & 0 & 0 & 1 & 0 & 1 & 0\\
	
	\end{bmatrix}
	\begin{bmatrix}
	b_0 \\
	b_1 \\
	b_2 \\
	b_3 \\
	b_4 \\
	b_5 \\
	b_6 \\
	b_7 \\	
	\end{bmatrix}
	+
	\begin{bmatrix}
	1 \\
	0 \\
	1 \\
	0 \\
	0 \\
	0 \\
	0 \\
	0 \\	
	\end{bmatrix}
	\]
	
\end{frame}

\begin{frame}[t]{InvSbox}
	\small
	The Inv-S-box used in the \textbf{InvSubBytes()} transformation is presented in hexadecimal for as in figure. \\
	For example, if $s_{1,1} = {53}$, then the substitution value would be determined by the intersection of the row with index '5' and the column with index '3'. This would result in $s^{'}_{1,1}$ having a value of ${50}$.
	
	\begin{table}[]
		\tiny 
		\setlength{\tabcolsep}{2pt} 
		\begin{tabular}{c|cccccccccccccccc}
			& 0    &    1 &    2 &    3 &    4 &    5 &    6 &    7 &    8 &    9 &    A &    B &  C &   D  &  E &  F \\ \hline
			0 & 0x52 & 0x09 & 0x6A & 0xD5 & 0x30 & 0x36 & 0xA5 & 0x38 & 0xBF & 0x40 & 0xA3 & 0x9E & 0x81 & 0xF3 & 0xD7 & 0xFB \\
			1 & 0x7C & 0xE3 & 0x39 & 0x82 & 0x9B & 0x2F & 0xFF & 0x87 & 0x34 & 0x8E & 0x43 & 0x44 & 0xC4 & 0xDE & 0xE9 & 0xCB \\
			2 & 0x54 & 0x7B & 0x94 & 0x32 & 0xA6 & 0xC2 & 0x23 & 0x3D & 0xEE & 0x4C & 0x95 & 0x0B & 0x42 & 0xFA & 0xC3 & 0x4E \\
			3 & 0x08 & 0x2E & 0xA1 & 0x66 & 0x28 & 0xD9 & 0x24 & 0xB2 & 0x76 & 0x5B & 0xA2 & 0x49 & 0x6D & 0x8B & 0xD1 & 0x25 \\
			4 & 0x72 & 0xF8 & 0xF6 & 0x64 & 0x86 & 0x68 & 0x98 & 0x16 & 0xD4 & 0xA4 & 0x5C & 0xCC & 0x5D & 0x65 & 0xB6 & 0x92 \\
			5 & 0x6C & 0x70 & 0x48 & 0x50 & 0xFD & 0xED & 0xB9 & 0xDA & 0x5E & 0x15 & 0x46 & 0x57 & 0xA7 & 0x8D & 0x9D & 0x84 \\
			6 & 0x90 & 0xD8 & 0xAB & 0x00 & 0x8C & 0xBC & 0xD3 & 0x0A & 0xF7 & 0xE4 & 0x58 & 0x05 & 0xB8 & 0xB3 & 0x45 & 0x06 \\
			7 & 0xD0 & 0x2C & 0x1E & 0x8F & 0xCA & 0x3F & 0x0F & 0x02 & 0xC1 & 0xAF & 0xBD & 0x03 & 0x01 & 0x13 & 0x8A & 0x6B \\
			8 & 0x3A & 0x91 & 0x11 & 0x41 & 0x4F & 0x67 & 0xDC & 0xEA & 0x97 & 0xF2 & 0xCF & 0xCE & 0xF0 & 0xB4 & 0xE6 & 0x73 \\
			9 & 0x96 & 0xAC & 0x74 & 0x22 & 0xE7 & 0xAD & 0x35 & 0x85 & 0xE2 & 0xF9 & 0x37 & 0xE8 & 0x1C & 0x75 & 0xDF & 0x6E \\
			A & 0x47 & 0xF1 & 0x1A & 0x71 & 0x1D & 0x29 & 0xC5 & 0x89 & 0x6F & 0xB7 & 0x62 & 0x0E & 0xAA & 0x18 & 0xBE & 0x1B \\
			B & 0xFC & 0x56 & 0x3E & 0x4B & 0xC6 & 0xD2 & 0x79 & 0x20 & 0x9A & 0xDB & 0xC0 & 0xFE & 0x78 & 0xCD & 0x5A & 0xF4 \\
			C & 0x1F & 0xDD & 0xA8 & 0x33 & 0x88 & 0x07 & 0xC7 & 0x31 & 0xB1 & 0x12 & 0x10 & 0x59 & 0x27 & 0x80 & 0xEC & 0x5F \\
			D & 0x60 & 0x51 & 0x7F & 0xA9 & 0x19 & 0xB5 & 0x4A & 0x0D & 0x2D & 0xE5 & 0x7A & 0x9F & 0x93 & 0xC9 & 0x9C & 0xEF \\
			E & 0xA0 & 0xE0 & 0x3B & 0x4D & 0xAE & 0x2A & 0xF5 & 0xB0 & 0xC8 & 0xEB & 0xBB & 0x3C & 0x83 & 0x53 & 0x99 & 0x61 \\
			F & 0x17 & 0x2B & 0x04 & 0x7E & 0xBA & 0x77 & 0xD6 & 0x26 & 0xE1 & 0x69 & 0x14 & 0x63 & 0x55 & 0x21 & 0x0C & 0x7D 
		\end{tabular}
	\end{table}
\end{frame}